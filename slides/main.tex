\documentclass{beamer}
\usetheme{metropolis} % Use metropolis theme
\usepackage[ngerman]{babel}

\title{Rolling-bouncing ball - Dynamik eines Balles}
\author{Helene Rößler, Norbert Hammer}
\date{}


\begin{document}
\maketitle

\section{Bewegungsgleichungen}

\begin{frame}{First Frame}
Hello, world!
\end{frame}

\section{Newmark-Verfahren}

\begin{frame}{Einführung}
Ausgangssituation ist eine Bewegungsgleichung der Form:
\begin{equation*}
	F = M \ddot{x} + C \dot{x} + Kx
\end{equation*}

\begin{itemize}
	\item $x(t) \in \mathbb{R}^{m}$: generalisierter Ortsvektor
	\item $M, C, K \in \mathbb{R}^{m \times m}$: Massen-, Dämpfungs- und Steifigkeitsmatrix
\end{itemize}

\end{frame}


\begin{frame}{Allgemeine Form}
	
\begin{itemize}
\item Ausgangswerte zum Zeitpunkt $t_n$
\begin{align*}
	x_n, \quad v_n := \dot{x}_n, \quad a_n := \ddot{x}_n.
\end{align*}
\item Näherung für $t_{n+1} = t_n + h$ ist allgemein gegeben durch:
\begin{align*}
	F_{n+1} &= M a_{n+1} + C v_{n+1} + K q_{n+1}\\
	v_{n+1} &= v_{n} + h\left({\left(1 - \gamma\right) a_{n} + \gamma{a_{n+1}}}\right)\\
	x_{n+1} &= x_{n} + h v_{n} + \frac{h^2}{2} \left({\left(1 - 2\beta\right) a_{n} + 2\beta a_{n+1}}\right)
\end{align*}
\end{itemize}

\end{frame}


\begin{frame}{Spezialfall}
	
\begin{itemize}
	\item  $\beta = \frac{1}{4}, \gamma = \frac{1}{2}$ $\Rightarrow$ quadratische Genauigkeit
	\item $K = \mathbf{0} \in \mathbb{R}^{m \times m}$  
\end{itemize}
Führt zu
\begin{align*}
	F(x_{n+1}) &= M a_{n+1} + C v_{n+1}\\
	v_{n+1} &= v_n + \frac{h}{2} (a_n + a_{n+1})\\
	x_{n+1} &= x_n + h v_n + \frac{h^2}{4} (a_n + a_{n+1}).
\end{align*}

\end{frame}

\begin{frame}{Setup für die ODE}

\textbf{Fall:} Ball befindet sich in der Flugphase, wir lösen eine ODE\\
\vspace{0.3cm}
\textbf{Annahmen:}
	\begin{itemize}
		\item Dämpfung: $C = \mathbf{0} \in \mathbb{R}^{2 \times 2}$
		\item Massenmatrix:
		\[
		M =
		\begin{pmatrix}
			m & 0 \\
			0 & m
		\end{pmatrix}
		\]
		\item Kraftvektor (Gravitation): \quad $F = m g$, \quad wobei $g \in \mathbb{R}^2$
		\item Anfangswerte: $x_0, v_0$ und $a_0$ werden vom Algorithmus direkt als aktueller Zustand des Balls übernommen.
	\end{itemize} 

\end{frame}

\begin{frame}{Setup für die DAE}
\textbf{Fall:} Ball befindet sich in der Rollphase, wir lösen eine DAE\\
\vspace{0.3cm}
\textbf{Annahmen:}
\begin{itemize}
	\item Dämpfung (durch Dämpfungsfaktor $c \in \mathbb{R}$) und Masse: 
	\begin{equation*}
		C :=
		\begin{pmatrix}
			c & 0 & 0\\
			0 & c & 0\\
			0 & 0 & 0
		\end{pmatrix},
		\quad
		M =
		\begin{pmatrix}
			m & 0 & 0 \\
			0 & m & 0 \\
			0 & 0 & 0
		\end{pmatrix}
	\end{equation*}
	\item Kraftvektor: 
	\[F = 
	\begin{pmatrix}
		mg + \lambda \nabla G(q)\\
		G(q)
	\end{pmatrix}.
	\]
\end{itemize} 

\end{frame}


\begin{frame}{Anfangswerte für die DAE}
	
\begin{itemize}
	\item $x_0$: aktuelle Position des Balls
	\item $v_0$: Tangentialgeschwindigkeit
	\item $a_0$: Formel für Beschleunigung entlang einer Kurve:
	\[
	a_0 = a \tau + \kappa v^2 \nu
	\]
	
	\qquad mit Tangentialvektor $\tau$, Normalvektor $\nu$ und\\
	\qquad Krümmung der Bahn $\kappa$: 
	
	\item kein Startwert für $\lambda$ benötigt
\end{itemize}

\end{frame}
	

\end{document}
